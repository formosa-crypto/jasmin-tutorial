\PassOptionsToPackage{table,dvipsnames}{xcolor}
\documentclass[
  xcolor={table,dvipsnames},
]{beamer}

\usetheme{metropolis}

\usepackage[T1]{fontenc}
\usepackage[english]{babel}

\newcommand{\hint}[1]{{\color{Red}{[HINT: #1]}}}
\newcommand{\nextframe}{\hint{NEXT FRAME}}

\begin{document}
\begin{frame}{Title page}
  In this tutorial we will give an overview of Jasmin, a programming language
  designed to write high-assurance high-speed cryptography.

  \vfill

  The tutorial will be pretty hands-on, so let's begin right away with an
  example.

  \nextframe
\end{frame}

\begin{frame}{memeq}
  Jasmin is a low-level programming language with C-like syntax.

  \vfill

  This is a simple function that checks if two memory regions \texttt{p}
  and \texttt{q} coincide in the first \texttt{n} quadwords.

  \vfill

  Jasmin comes with tools to verify that code is
  \begin{itemize}
  \itemsep=1em
  \item[] correct (relative to a specification);
  \item[] safe (e.g., it doesn't divide by zero); and
  \item[] secure (e.g., against side-channel attacks).
  \end{itemize}
\end{frame}

\begin{frame}{memeq}
  It has assignments, whiles, ifs, and functions.

  \vfill

  But is is lower-level than C: it really is structured assembly.

  \vfill

  Each Jasmin instruction must correspond to an assembly instruction.

  \nextframe
\end{frame}

\begin{frame}{memeq 2}
  Indeed, we can use assembly instructions directly in the Jasmin source file.

  \nextframe

  \vfill

  Here I'm using the \texttt{INC} x86-64 instruction to increment \texttt{i},
  this is the dashed red arrow.
\end{frame}

\begin{frame}{memeq 2}
  There are two reasons we need a language at a lower-level than C: efficiency
  and security.

  \vfill

  Regarding efficiency, we can be much more precise and detailed with our
  optimizations.

  \vfill

  Regarding security, we can see many low-level problems at source-level,
  without the compiler getting in the way.
\end{frame}


\begin{frame}{memeq 2}
  Well, but then why bother with Jasmin when we have assembly?
  Same reasons, different emphasis: security and efficiency.

  \vfill

  A structured language with a clearly defined semantics avoids tons of the
  problems of trusting large assembly codebases.

  \vfill

  It is also important for efficiency: higher-assurances on our code allows more
  aggressive optimizations.
\end{frame}

\begin{frame}{memeq 2}
  Jasmin gives structure to assembly programs with functions, conditionals and
  loops without compromising efficiency or security.

  \nextframe

  Their compilation is standard and predictable:
  \begin{itemize}
  \itemsep=1em
  \item[] functions (in dotted green) compile to a label and a return;
  \item[] loops (in continuous red) compile to a check and a backward jump; and
  \item[] conditionals (in dashed blue) compile to a check and a jump to the
  else branch.
  \end{itemize}
\end{frame}

\begin{frame}{More online}
  Now let's write some Jasmin.

  \hint{Have people open the files and setup Docker.}

  \hint{Leave the Formosa slide while people work.}
\end{frame}

\begin{frame}{gimli}
  In \texttt{jazz\_gimli.h} you can see the functions we will implement today.

  \vfill

  In \texttt{gimli.jazz} you have the function declarations with empty bodies
  and some hints.

  \vfill

  These functions use pointers, which have the same syntax as C arrays.
  \hint{Show \texttt{otp\_fixed} from \texttt{otp.jazz}.}
\end{frame}

\begin{frame}{gimli: sboxes}
  Operators are basically the same as in C, for instance here addition is plus
  and exclusive or is caret.

  \vfill

  \hint{After we're done with sboxes.}

  The next functions use control flow and arrays, which are also similar to C,
  except for for loops.

  \vfill

  The syntax is \texttt{for x = N to M} or \texttt{for x = N downto M}, and the
  loop is always unrolled.

  \vfill

  This means that the loop range needs to be known at compile time.
\end{frame}

\begin{frame}{otp}
  \hint{Show \texttt{otp.jazz}}

  This function takes three pointers, XORs the message and the key and writes
  the result to the ciphertext.

  Array accesses have the same syntax as in C.

  \vfill

  For loops in Jasmin are always unrolled, this means that they need to have
  constant ranges.

  This function will have \(16 \times 3 = 48\) instructions.
\end{frame}

\begin{frame}{EasyCrypt}
  One of the reasons to use Jasmin is that we can formally prove things about
  programs in the EasyCrypt proof assistant.

  \vfill

  \hint{Show terminal with \texttt{jasminc otp.jazz -ec otp}.}

  Here I extract a representation of the otp function which I can paste in an EC
  file.

  \vfill

  \hint{Show EasyCrypt file.}

  The proof assistant lets us write statements like ``if the key comes from a
  uniform distribution, the output is uniformly distributed.''
\end{frame}

\begin{frame}{gimli}
  \hint{Back to gimli.}
\end{frame}

\begin{frame}{safety}
  Show how to avoid common pitfalls.
\end{frame}

\begin{frame}{side channels}
  Show the CT checker and the SCT checker.
\end{frame}

\begin{frame}{gimli: AVX}
  Jasmin supports vectorized (AVX) instructions.
\end{frame}

\end{document}
