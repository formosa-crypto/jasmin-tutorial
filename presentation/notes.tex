\PassOptionsToPackage{table,dvipsnames}{xcolor}
\documentclass[
  xcolor={table,dvipsnames},
]{beamer}

\usetheme{metropolis}

\usepackage[T1]{fontenc}
\usepackage[english]{babel}

\newcommand{\hint}[1]{{\color{Red}{[HINT: #1]}}}
\newcommand{\nextframe}{\hint{NEXT FRAME}}

\begin{document}
\begin{frame}{Title page}
  In this tutorial we will give an overview of Jasmin, a programming language
  designed to write high-assurance high-speed cryptography.

  \vfill

  We are Miguel and Santiago from MPI-SP.

  \vfill

  The tutorial will be pretty hands-on, so let's begin right away with an
  example.

  \nextframe
\end{frame}

\begin{frame}{memeq}
  Jasmin is a low-level programming language with C-like syntax.

  \vfill

  This is a simple function that checks if two memory regions \texttt{p}
  and \texttt{q} coincide in the first \texttt{n} quadwords.

  \vfill

  Jasmin comes with tools to verify that code is
  \begin{itemize}
  \itemsep=1em
  \item[] correct (relative to a specification);
  \item[] safe (e.g., it doesn't divide by zero); and
  \item[] secure (e.g., against side-channel attacks).
  \end{itemize}
\end{frame}

\begin{frame}{memeq}
  It has assignments, whiles, ifs, and functions.

  \vfill

  But is is lower-level than C: it really is structured assembly.

  \vfill

  Each Jasmin instruction must correspond to an assembly instruction.

  \nextframe
\end{frame}

\begin{frame}{memeq 2}
  Indeed, we can use assembly instructions directly in the Jasmin source file.

  \nextframe

  \vfill

  Here I'm using the \texttt{INC} x86-64 instruction to increment \texttt{i},
  this is the dashed red arrow.
\end{frame}

\begin{frame}{memeq 2}
  There are two reasons we need a language at a lower-level than C: efficiency
  and security.

  \vfill

  Regarding efficiency, we can be much more precise and detailed with our
  optimizations.

  \vfill

  Regarding security, we can see many low-level problems at source-level,
  without the compiler getting in the way.
\end{frame}


\begin{frame}{memeq 2}
  Well, but then why bother with Jasmin when we have assembly?
  Same reasons, different emphasis: security and efficiency.

  \vfill

  A structured language with a clearly defined semantics avoids tons of the
  problems of trusting large assembly codebases.

  \vfill

  It is also important for efficiency: higher-assurances on our code allows more
  aggressive optimizations.
\end{frame}

\begin{frame}{memeq 2}
  Jasmin gives structure to assembly programs with functions, conditionals and
  loops without compromising efficiency or security.

  \nextframe

  Their compilation is standard and predictable:
  \begin{itemize}
  \itemsep=1em
  \item[] functions (in dotted green) compile to a label and a return;
  \item[] loops (in continuous red) compile to a check and a backward jump; and
  \item[] conditionals (in dashed blue) compile to a check and a jump to the
  else branch.
  \end{itemize}
\end{frame}

\begin{frame}{More online}
  Now let's write some Jasmin.

  \hint{Have people open the files and setup Docker.}

  \hint{Leave the Formosa slide while people work.}
\end{frame}

\begin{frame}{jazz\_gimli.h}
  In this file you can see the functions we will implement today.

  In \texttt{gimli.jazz} you have the function declarations with empty bodies.
\end{frame}

\begin{frame}{otp}
  A bit more on syntax: operators are almost the same as in C, for instance here
  addition is plus and exclusive or is caret.

  \vfill

  Functions marked \texttt{export} are the only entry points of the program.
  They are what is visible from C (i.e., by the linker).

  \vfill

  Raw memory addresses use square brackets instead of asterisk, and you can
  specify the size of the load by prefixing it
  with \texttt{u8}, \texttt{u16}, \texttt{u32}, \texttt{u64}, and so on.

  The default load size is register size.
\end{frame}

\begin{frame}{otp\_fixed}
  Array accesses have the same syntax as in C.
\end{frame}

\begin{frame}{gimli: sboxes}
  Until we're done with sboxes.
\end{frame}

\begin{frame}{safety}
  Show how to avoid common pitfalls.
\end{frame}

\begin{frame}{gimli: structure}
  Some time?
\end{frame}

\begin{frame}{side channels}
  Show the CT checker and the SCT checker.
\end{frame}

\begin{frame}{gimli: finish \& AVX}
\end{frame}

\end{document}
